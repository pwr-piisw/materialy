\documentclass[12pt]{article}
\usepackage[utf8]{inputenc}
\usepackage{polski}
\usepackage[a4paper, left=2.0cm, right=2.0cm, top=2.0cm, bottom=2.0cm]{geometry}
\usepackage{hyperref}
\usepackage{graphicx}

\title{PIISW, W-4, IO, 2021/2022, semestr letni\\Lista zadań nr~1: HTML i~CSS}
\author{mgr inż. Maciej Małecki\\ \small maciej.malecki@pwr.edu.pl}

\begin{document}
    \maketitle

    \section*{Zasady oddawania zadań}
        \begin{enumerate}
            \item Zadania z~tej listy mogą być oddawane \emph{wyłącznie} za pośrednictwem prywatnego repozytorium na portalu \texttt{github.com}.
            \item Przed zajęciami, na których oddawana będzie lista należy nadać prowadzącemu uprawnienia do odczytu dla w.w. repozytorium.
            \item Rozwiązanie każdego z~zadań musi znaleźć się w~katalogu o~nazwie \texttt{zad-x} gdzie \texttt{x} jest numerem zadania.
            \item Rozwiązanie każdego z~zadań musi mieć nazwę \texttt{index.html}.
            \item Nie wolno używać żadnych gotowych bibliotek styli (Bootstrap, itd).
            \item Każde rozwiązanie powinno działać po otwarciu w.w. pliku w~przeglądarce Chrome (chyba, że w~zadaniu zaznaczono inaczej).
        \end{enumerate}

    \section*{Oceny}
    \begin{tabular}{|l|c|c|c|c|c|c|}
        \hline
        Punkty: & $<9$ & $9-10$ & $11-12$ & $13-14$ & $15-16$ & $17-18$\\
        \hline
        Ocena:  & $2,0$ & $3,0$ & $3,5$ & $4,0$ & $4,5$ & $5,0$\\
        \hline
    \end{tabular}

    \section*{Zadania}
    \begin{enumerate}
        \item\label{exc:basic-styling}
            (5 pkt) Utwórz dokument HTML~5 wraz z~arkuszem styli zawierającym wszystkie z~następujących elementów semantycznych: \texttt{article}, \texttt{section}, \texttt{p}, \texttt{h1}, \texttt{h2}, \texttt{h3} (wypełnij paragrafy treścią, np. lorem ipsum), który spełnia następujące warunki:
            \begin{enumerate}
                \item krój pisma dla całej strony powinien być bezszeryfowy,
                \item maksymalna szerokość tekstu dokumentu powinna wynosić 1024 piksele,
                \item w przypadku monitorów o~większej szerokości biały obszar z~z tekstem powinien wyświetlać się na środku ekranu, po bokach powinny pojawić się jasnoszare marginesy o~równej szerokości,
                \item odstępy między paragrafami powinny wynosić 2em,
                \item odstępy między rozdziałami powinny wynosić 3em,
                \item boczne marginesy tekstu powinny wynosić 3em,
                \item górny i~dolny margines tekstu powinien wynosić 2em,
                \item dla wszystkich elementów występujących na stronie ustaw właściwość \texttt{box-sizing} na \texttt{border-box}.
            \end{enumerate}

            Elementy HTML muszą być użyte zgodnie z~ich semantyką.

            Patrz też rysunek~\ref{fig:basic-styling}.

            \begin{figure}[p]
                \centering
                \includegraphics[width=0.6\textwidth]{lista-1-1}
                \caption{Przykładowa konstrukcja dokumentu dla zadania~\ref{exc:basic-styling}.}
                \label{fig:basic-styling}
            \end{figure}

        \item\label{exc:layout}
            (7 pkt) Utwórz dokument HTML~5 wraz z~arkuszem styli, który składa się ze znajdującego się zawsze na górze strony elementu tytułowego, ze spisu treści znajdującego się po lewej stronie oraz z~zawartości (lorem ipsum) zajmującej zawsze całą pozostałą część ekranu (patrz rysunek~\ref{fig:layout}).

            Następujące warunki muszą być spełnione:
            \begin{enumerate}
                \item obszar artykułu powinien się automatycznie dostosowywać do przeskalowywanego okna przeglądarki tak, aby zajmować całe dostępne miejsce,
                \item pionowy pasek przewijania powinien pojawić jedynie gdy zawartość artykułu nie mieści się w~obszarze,
                \item pionowy pasek przewijania powinien przesuwać jedynie zawartość artykułu, a~nie całą stronę,
                \item pionowy pasek przewijania całej strony nigdy nie powinien być widoczny,
                \item wymiary pokazane na rysunku~\ref{fig:layout} muszą być zachowane,
                \item minimalna szerokość okna przeglądarki powinna wynosić 240px (poniżej tej wartości powinien pojawić się pasek przewijania poziomego okna przeglądarki).
            \end{enumerate}

            \begin{figure}[p]
                \centering
                \includegraphics[width=0.7\textwidth]{lista-1-2}
                \caption{Wymagany układ strony dla zadania~\ref{exc:layout}.}
                \label{fig:layout}
            \end{figure}

        \item\label{exc:button-with-icon}
            (6 pkt) Utwórz dokument zawierający panel trzech przycisków: ``Go back'', ``Save'', ``Delete'' (elementy typu \texttt{button}). Każdy przycisk powinien składać się z~odpowiedniej ikony oraz etykiety. Należy usunąć wszelkie natywne style, jakie przeglądarka (Chrome) nakłada domyślnie na elementy \texttt{button}.

            Najlepiej użyć ikon pochodzących z~biblioteki Font Awesome lub Line Awesome - odpowiednią bibliotekę proszę ściągnąć, rozpakować i~umieścić w~repozytorium.

            Dodatkowo muszą być spełnione następujące warunki:
            \begin{enumerate}
                \item przyciski muszą mieć ustaloną, jednakową szerokość;
                \item przyciski muszą być umieszczone w~kontenerze, który zapewnia automatyczne ich rozmieszczenie w~dwóch lub trzech wierszach, gdy brakuje miejsca;
                \item przyciski powinny zajmować miejsce począwszy od lewej strony, z~odstępem 10px między nimi;
                \item ramka przycisku powinna być niebieskawa, rogi powinnny być zaokrąglone;
                \item po najechaniu na przycisk kursorem myszy, kształt kursora powinien się zmienić, zmianie powinien ulegać też kolor etykiety i~ikony, nie powinno zmieniać się za to obramowanie;
                \item przycisk wybrany np. poprzez klawisz ``TAB'' powinien być zaznaczony cieniem.
            \end{enumerate}

            Wszystkie szczegóły zostały opisane i~pokazane na ilustracji~\ref{fig:button-with-icon}.

            \begin{figure}[p]
                \centering
                \includegraphics[width=0.9\textwidth]{lista-1-3}
                \caption{Wymagany układ strony dla zadania~\ref{exc:button-with-icon}.}
                \label{fig:button-with-icon}
            \end{figure}
    \end{enumerate}
\end{document}
