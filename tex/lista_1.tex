\documentclass[12pt]{article}
\usepackage[utf8]{inputenc}
\usepackage{polski}
\usepackage[a4paper, left=2.0cm, right=2.0cm, top=2.0cm, bottom=2.0cm]{geometry}
\usepackage{hyperref}

\title{PIISW, W08, IO, 2019/2020, semestr letni\\Lista zadań nr 1}
\author{mgr inż. Maciej Małecki\\ \small maciej.malecki@pwr.edu.pl}

\begin{document}
    \maketitle

    \section*{Wprowadzenie}
        Do realizacji zadań z~tej listy niezbędne jest wykonanie zadań z~listy~0, w~szczególności zaś konieczne jest posiadanie konta na portalach \texttt{github.com} oraz \texttt{circleci.com}.

        Rozwiązanie każdego z~zadań musi znaleźć się na prywatnym repozytorium studenta (repozytorium takie tworzone jest w~ramach zadania~\ref{exc:git-repo}).

    \section*{Oceny}
    \begin{tabular}{|l|c|c|c|c|c|c|}
        \hline
        Punkty: & $<8$ & $8-9$ & $10-11$ & $12-13$ & $14-15$ & $16-17$\\
        \hline
        Ocena:  & $2,0$ & $3,0$ & $3,5$ & $4,0$ & $4,5$ & $5,0$\\
        \hline
    \end{tabular}

    \section*{Zadania}
    \begin{enumerate}
        \item\label{exc:github-access}
            (2 pkt) Konfiguracja dostępu do serwisu \texttt{github.com}.
            \begin{enumerate}
                \item Wygeneruj parę kluczy publicznych/prywatnych o~długości 4096 bitów (z użyciem narzędzia \texttt{ssh-keygen} lub \texttt{putty-keygen}\footnote{Klucz w~formacie putty należy przekonwertować do formatu OpenSSH.}).
                \item Zarejestruj wygenerowany klucz publiczny w~swoim profilu w serwisie github.com.
                \item Skonfiguruj lokalnego klienta GIT tak, aby dla domeny \texttt{github.com} użyty był wygenerowany klucz prywatny\footnote{Skorzystaj z~informacji zawartych w~dokumencie \texttt{GIT Cheat Sheet.pdf} dostępnym pod adresem \url{https://pwr-piisw.github.io/materialy}.
            \end{enumerate}

        \item\label{exc:git-repo}
            (2 pkt) Praca z lokalną kopią repozytorium.
            \begin{enumerate}
                \item Utwórz prywatne repozytorium w~ramach swojego konta na GitHubie.
                \item Wykorzystując funkcjonalność portalu \texttt{github} zaimportuj zawartość repozytorium \url{https://github.com/pwr-piisw/oasp4js-ng-boot-project-seed.git} do swojego repozytorium prywatnego\footnote{Możliwe jest także użycie \texttt{git remote add} oraz \texttt{git push} aby zaimportować zawartość repozytorium w~przypadku, gdy repozytorium prywatne już istnieje i~jest puste.}.
                \item Nadaj uprawnienia (zapis/odczyt) do repozytorium prowadzącemu.
                \item Sklonuj zawartość repozytorium z~użyciem lokalnego klienta GIT.
                \item Zmodyfikuj zawartość pliku \texttt{README.adoc} dodając nowy paragraf tekstu.
                \item Zatwierdź (ang. \textit{commit}) i~skomentuj zmianę, przenieś zmianę (ang. \textit{push}) do repozytorium zdalnego.
            \end{enumerate}

        \item\label{exc:branches}
            (3 pkt) Praca z~gałęziami oraz \textit{pull request}'ami.
            \begin{enumerate}
              \item \label{exc:branches-git-branch}Utwórz \textit{branch} oraz \textit{wycheckout}'uj go. Wprowadź zmianę w~kodzie na branchu i~zatwierdź ją. Wypushuj branch.
              \item Przejdź na \textit{branch} \texttt{master}. Wprowadź inną zmianę, zatwierdź ją oraz wypushuj.
              \item W~serwisie \texttt{github.com} stwórz \textit{pull request} dla brancha utworzonego w~punkcie~\ref{exc:branches-git-branch}. W~przypadku braku konfliktów, \texttt{github} zaoferuje trzy możliwości scalenia zmian. Wybierz opcję: \textit{Rebase and merge}.
            \end{enumerate}

        \item\label{exc:squash}
            (4 pkt) Porządkowanie historii.
            \begin{enumerate}
              \item \label{exc:squash-git-branch}Utwórz kolejny \textit{branch} oraz \textit{wyecheckout}'uj go. Dokonaj na tym \textit{branchu} trzy zmiany każdą niezależnie \textit{commitując}. \textit{Wypushuj} branch na zdalne repozytorium.
              \item Wróć na gałąź \texttt{master} oraz przenieś zmiany z~gałęzi utworzonej w~punkcie~\ref{exc:squash-git-branch} w~taki sposób, aby były one widoczne jako jeden commit nie będący merge-commitem. \textit{Wypushuj} zawartość brancha \texttt{master}, pozostaw także \textit{commity} na \textit{branchu} z~punktu~\ref{exc:squash-git-branch}.
            \end{enumerate}
            \textbf{Wskazówka:} zadanie można wykonać na kilka sposobów, warto poszukać w~sieci informacji na temat techniki \textit{git squash}.

        \item\label{exc:travis-ci}
            (2 pkt) Konfiguracja środowiska \textit{continuous integration}.
            \begin{enumerate}
              \item Utwórz konto w~serwisie \texttt{circleci.com} oraz skojarz je ze swoim kontem w~serwisie \texttt{github.com}.
              \item Aktywuj środowisko CI dla repozytorium utworzonego w~zadaniu~\ref{exc:git-repo}, uruchom ręcznie pierwszy \textit{build}.
              \item Wprowadź kolejną zmianę do pliku \texttt{README.adoc}, zatwierdź i~\textit{wypushuj}, sprawdź czy kolejny \textit{build} uruchamia się automatycznie.
              \item Zaktualizuj link do grafiki reprezentującej status \textit{builda} tak, aby wskazywał na właściwe środowisko w~ramach \texttt{CircleCI}.
            \end{enumerate}

        \item\label{exc:maven}
            (4 pkt) W~systemie obsługującym bibliotekę istnieją trzy komponenty: komponent zarządzający użytkownikami, komponent odpowiedzialny za książki oraz komponent w~którym znajduje się logika i~reguly biznesowe procesów wypożyczania książek. Komponent wypożyczania zależy od komponentu obsługi książek, a~ten z~kolei zależy od komponentu obsługi użytkownika.
            \begin{enumerate}
                \item\label{exc:maven-directory} Utwórz podkatalog w~repozytorium, pozostałe punkty tego zadania należy wykonać w~tym katalogu lub jego podkatalogach.
                \item Przygotuj komendy Maven’owe po których uruchomieniu powstanie trójmodułowy projekt zgodny z~ww. zależnościami.
                \item Główny plik \texttt{pom.xml} powinien znajdować się w~katalogu~\ref{exc:maven-directory}, a~każdy z~modułów powinien mieć swój podkatalog.
            \end{enumerate}

    \end{enumerate}
\end{document}

