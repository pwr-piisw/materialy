\documentclass[12pt]{article}
\usepackage[utf8]{inputenc}
\usepackage{polski}
\usepackage[a4paper, left=2.0cm, right=2.0cm, top=2.0cm, bottom=2.0cm]{geometry}
\usepackage{hyperref}
\usepackage{skull}

\title{PIISW, W4N, IO, 2021/2022, semestr letni\\Lista zadań nr 5: Angular}
\author{Maciej Małecki\\ \small maciej.malecki@pwr.edu.pl}

\begin{document}
    \maketitle

    \section*{Wprowadzenie}
        \begin{enumerate}
            \item W~celu realizacji listy należy założyć nowe prywatne repozytorium oraz zainicjować je zawartością z~\href{https://github.com/pwr-piisw/bookstore2022}{https://github.com/pwr-piisw/bookstore2022}.
            \item Backend oferujący REST API został zaimplementowany z~wykorzystaniem \texttt{json-server} i~może być wystartowany z~linii poleceń w~następujący sposób: \texttt{npm run backend:start}.
			\item Plik \texttt{README.md} zawiera wszelkie informacje potrzebne do zbudowania i~uruchomienia aplikacji.
        \end{enumerate}

    \section*{Oceny}
    \begin{tabular}{|l|c|c|c|c|c|c|c|}
        \hline
        Punkty: & $<9$ & $9-10$ & $11-12$ & $13-14$ & $15-16$ & $17-18$ & $>18$\\
        \hline
        Ocena:  & $2,0$ & $3,0$ & $3,5$ & $4,0$ & $4,5$ & $5,0$ & $5,5$\\
        \hline
    \end{tabular}

    \section*{Zadania}
    \begin{enumerate}

		\item\label{book-details}
			(3 pkt) W~aplikacji Bookstore zaimplementowano widok listy książek (dostępny pod linkiem \texttt{/books}).
			Zaimplementuj widok szczegółowy dla książki prezentujący tytuł, autora, rok wydania oraz opis książki.

			\begin{enumerate}
				\item W~serwisie \texttt{BooksService} zaimplementuj metodę \texttt{findBookById} i~zintegruj ją z~odpowiednim endpointem w~backendzie.

				Przykład URL endpointu: \href{http://localhost:3000/api/books/1}{http://localhost:3000/api/books/1}.

				\item Widok szczegółowy powinien być zaimplementowany jako nowy komponent modułu \texttt{books}.

				\item Dodaj routing \texttt{/books/:bookId/reviews}\footnote{Zwróć uwagę, że \texttt{:bookId} oznacza w~tym kontekście zmienną część URL.}, dane do widoku wczytaj przy pomocy resolvera.

				Routing powinien wspierać tzw. \emph{deep linking}, tj. powinna być możliwość wejścia bezpośrednio na widok szczegółowy po podaniu adresu do niego w~polu adresu przeglądarki.

				\item Link \texttt{Reviews} elementu listy książek powinien przekierowywać do widoku szczegółowego.

				\item W~widocznym miejscu na widoku szczegółowym umieść link pozwalający na powrót do widoku listy.
			\end{enumerate}

		\item
			(6 pkt) Zaimplementuj widok edycji dla książki.

			Widok edycji powinien składać się formularza oraz pól edycyjnych pozwalających na zmianę każdego pola obiektu \emph{Book} za wyjątkiem identyfikatora (\texttt{id}).
			Opis książki zrealizuj przy pomocy edytora \texttt{<textarea>}, pozostałe pola przy pomocy edytorów typu \texttt{<input>}.
			Wykorzystaj mechanizm reaktywnych formularzy (\emph{reactive forms}).

			\begin{enumerate}
				\item Dodaj routing \texttt{/books/:bookId/edit}, dane do widoku \texttt{book-details} wczytaj przy pomocy resolvera\footnote{Zwróć uwagę, że można użyć identycznego resolvera jak w~zadaniu~\ref{book-details}.}.
				Routing powinien wspierać \emph{deep linking}.

				\item Link \texttt{Edit} elementu listy książek powinien przekierowywać do widoku edycji.

				\item Zaimplementuj walidację pól edycyjnych:
				\begin{enumerate}
					\item tytuł książki jest obowiązkowy i~nie dłuższy niż 50 znaków,
					\item autor jest obowiązkowy i~nie dłuższy niż 50 znaków,
					\item rok wydania jest liczbą z~zakresu $1000-2023$,
					\item opis jest nieobowiązkowy i~nie dłuższy niż 1000 znaków.
				\end{enumerate}
				Błędne pola powinny być podświetlone na czerwono.

				\item Zaimplementuj przycisk "Save".

				Przycisk ten powinien być nieaktywny w~przypadku wystąpienia jakiegokolwiek błędu walidacji a~także wtedy, gdy zawartość formularza edycyjnego nie została jeszcze przez użytkownika zmieniona.

				Przycisk po naciśnięciu powinien spowodować zapis zmodyfikowanej książki oraz przekierowanie na widok listy książek.
				Zmiana powinna być widoczna na liście książek.

				\item Zaimplementuj przycisk "Cancel".

				Przycisk ten powinien być zawsze aktywny.

				Przycisk po naciśnięciu powinien przekierować na listę książek, edytowana książka nie powinna być zapisana.
			\end{enumerate}

		\item
			(4 pkt) Rozszerz widok szczegółowy o~listę recenzji. Wyświetl wszystkie recenzje jedna pod drugą. Każda recenzja powinna zawierać autora, tytuł, treść oraz ocenę.
			\begin{enumerate}
				\item Wyświetlanie recenzji powinno być zrealizowane przy użyciu osobnego komponentu (jeden komponent na recenzję, użyj \texttt{*ngFor} do wyświetlenia listy).
				\item Zaimplementuj osobny serwis do wczytywania recenzji (należy w~tym celu stworzyć osobny resolver).

				Zauważ, że \texttt{json-server} pozwala na odpytanie o~recenzje dla danej książki z~wykorzystaniem \href{http://localhost:3000/api/reviews?forBookId=1}{http://localhost:3000/api/reviews?forBookId=1}.
				\item Zintegruj wczytywanie recenzji z~widokiem widoku szczegółowego.
			\end{enumerate}

		\item
			(5 pkt) Dodaj możliwość dodawania nowej recenzji.
			\begin{enumerate}
				\item Stwórz komponent edycyjny z~polami formularza pozwalającymi na podanie autora, tytułu i~treści.

				Komponent edycyjny można zrealizować zarówno jako osobny widok lub też pokazać go na widoku \texttt{book-details}.
				\item Wszystkie pola powinny być obowiązkowe - zaimplementuj odpowiednią walidację.
				\item Rozszerz serwis dla recenzji o funkcjonalność zapisu recenzji.
				\item Po zatwierdzeniu recenzji (przy użyciu np. przycisku ``Save'') recenzja powinna być zapisana w~backendzie, a~widok szczegółowy odpowiednio zaktualizowany.
			\end{enumerate}

		\item
			$\skull$(6 pkt) Wyszukiwanie pełnotekstowe książek.

			Na widoku listy książek umieść pole edycyjne pozwalające na wpisywanie tekstu.
			Po wpisaniu co najmniej dwóch znaków lista książek powinna zostać zawężona do tych, które zawierają w~swoich polach podany tekst.

			\begin{enumerate}
				\item Zauważ, że \texttt{json-server} pozwala na wykonania wyszukania pełnotekstowego dzięki następującemu zapytaniu: \href{http://localhost:3000/api/books?q=query}{http://localhost:3000/api/books?q=query}.

				\item Wyszukiwanie powinno zostać wykonane niezwłocznie po zakończeniu pisania przez użytkownika.

				\item Lista książek powinna zostać zaktualizowana (odpowiednio zawężona) bez przeładowywania strony.
			\end{enumerate}

    \end{enumerate}
\end{document}

